\documentclass[12pt]{article}

\begin{document}

\title{Nuclear Medicine and Radiation Protection}
\author{Dinesh Saini (18111022)}
\date{January 21, 2022}
\maketitle



Many disorders require the use of nuclear medicine and molecular imaging technologies for diagnosis, assessment, and treatment. These include heart, bone, brain, and renal problems, as well as oncology applications. The technology underpinning many nuclear medicine diagnostic and therapeutic applications is discussed in the following article. Finally, the topic of radiation safety in nuclear medicine will be examined. It's critical that every member of the nuclear medicine team be prepared to talk to our patients and their families about the benefits and hazards of nuclear medicine.
\par
The delivery of a small amount of a radioactive tracer or radiopharmaceutical to the patient allows health care practitioners to analyse molecular and physiologic processes within the body in these highly effective, safe, and painless approaches. For over 60 years, these tests have been utilised to assess virtually every human system, including the heart and brain, as well as to scan many types of cancer. In addition to diagnostic imaging, radiopharmaceuticals can be given to patients for therapeutic purposes, such as the treatment of thyroid illness with radioactive iodine. This article will only cover diagnostic nuclear medicine methods and will not cover therapeutic elements.


\end{document}