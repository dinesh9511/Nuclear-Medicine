\documentclass[12pt]{article}

\begin{document}

\title{Nuclear Medicine and Radiation Protection}
\author{Dinesh Saini (18111022)}
\maketitle



Many disorders require the use of nuclear medicine and molecular imaging technologies for diagnosis, assessment, and treatment. These include heart, bone, brain, and renal problems, as well as oncology applications. The technology underpinning many nuclear medicine diagnostic and therapeutic applications is discussed in the following article. Finally, the topic of radiation safety in nuclear medicine will be examined. It's critical that every member of the nuclear medicine team be prepared to talk to our patients and their families about the benefits and hazards of nuclear medicine.
\par
The delivery of a small amount of a radioactive tracer or radiopharmaceutical to the patient allows health care practitioners to analyse molecular and physiologic processes within the body in these highly effective, safe, and painless approaches. For over 60 years, these tests have been utilised to assess virtually every human system, including the heart and brain, as well as to scan many types of cancer. In addition to diagnostic imaging, radiopharmaceuticals can be given to patients for therapeutic purposes, such as the treatment of thyroid illness with radioactive iodine. This article will only cover diagnostic nuclear medicine methods and will not cover therapeutic elements.
\par
Nuclear medicine is unique in that it provides information about a patient's condition that is difficult or impossible to gather using conventional diagnostic imaging techniques. This is because, rather than focusing solely on anatomy and structure, nuclear medicine analyses function, rates of metabolism, and several other physiological activities within the body. Functional changes occur long before physical abnormalities arise or become obvious in many disease conditions. As a result, nuclear medicine can frequently provide vital information to clinicians, such as early detection and degree of disease, whether the disease is developing, and whether or not a current treatment is effective.
\par
The delivery of a modest amount of a radioactive tracer to scan certain physiological processes of certain organs is used in diagnostic nuclear medicine procedures. These strategies aid in early illness identification, patient management, and therapeutic decisions, and provide an important tool for monitoring medication effectiveness or assessing disease development. Nuclear medicine has several clinical uses that are advantageous. Myocardial perfusion imaging, tumour imaging, bone metabolism imaging, regional and relative renal function detection, and brain imaging to detect and locate epileptic foci and determine the activity of brain tumours are only a few examples.
\par
The tracer travels to the location of the body of interest once the radiopharmaceutical is administered, which is usually by intravenous injection. Different radiopharmaceuticals will target various tissues.
A radiopharmaceutical is made up of a radionuclide and a pharmaceutical component. The radionuclide component of the tracer produces radiation in the form of gamma rays, beta particles, or positrons. 99mTc and 123I, for example, mostly release gamma rays, whereas 18F emits positrons. These tracers have short half-lives (99mTc and 18F have half-lives of 6 and 2 hours, respectively), allowing adequate time to complete the investigation while minimising patient radiation exposure.The pharmaceutical component of the tracer determines where the radiopharmaceutical will accumulate in the body. MDP (99mTc medronate) concentrates in the skeleton, whereas DMSA (99mTc dimercaptosuccinic acid) concentrates in the kidneys. 18F fluorodeoxyglucose (FDG) is a glucose analogue that is metabolised similarly to glucose in the body. In some circumstances, such as 99mTc mercaptoacetyltriglycine (MAG3) kidney and 99mTc-labeled red blood cells studies, imaging occurs right after the radiotracer is administered.Other radiotracers, such as 99mTc DMSA, 99mTc MDP, and FDG, require waiting periods (ranging from 1 to 3 hours) between the time the tracer is administered and the time the pictures are taken to allow the agent to disperse to the tissue of interest and clear from the blood or nearby tissues. Initial imaging and further imaging may be required in some circumstances, such as a 99mTc MDP bone scan.


\end{document}