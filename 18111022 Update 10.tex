\documentclass[12pt]{article}

\begin{document}

\title{Nuclear Medicine and Radiation Protection}
\author{Dinesh Saini (18111022)}
\maketitle



Many disorders require the use of nuclear medicine and molecular imaging technologies for diagnosis, assessment, and treatment. These include heart, bone, brain, and renal problems, as well as oncology applications. The technology underpinning many nuclear medicine diagnostic and therapeutic applications is discussed in the following article. Finally, the topic of radiation safety in nuclear medicine will be examined. It's critical that every member of the nuclear medicine team be prepared to talk to our patients and their families about the benefits and hazards of nuclear medicine.
\par
The delivery of a small amount of a radioactive tracer or radiopharmaceutical to the patient allows health care practitioners to analyse molecular and physiologic processes within the body in these highly effective, safe, and painless approaches. For over 60 years, these tests have been utilised to assess virtually every human system, including the heart and brain, as well as to scan many types of cancer. In addition to diagnostic imaging, radiopharmaceuticals can be given to patients for therapeutic purposes, such as the treatment of thyroid illness with radioactive iodine. This article will only cover diagnostic nuclear medicine methods and will not cover therapeutic elements.
\par
Nuclear medicine is unique in that it provides information about a patient's condition that is difficult or impossible to gather using conventional diagnostic imaging techniques. This is because, rather than focusing solely on anatomy and structure, nuclear medicine analyses function, rates of metabolism, and several other physiological activities within the body. Functional changes occur long before physical abnormalities arise or become obvious in many disease conditions. As a result, nuclear medicine can frequently provide vital information to clinicians, such as early detection and degree of disease, whether the disease is developing, and whether or not a current treatment is effective.
\par
The delivery of a modest amount of a radioactive tracer to scan certain physiological processes of certain organs is used in diagnostic nuclear medicine procedures. These strategies aid in early illness identification, patient management, and therapeutic decisions, and provide an important tool for monitoring medication effectiveness or assessing disease development. Nuclear medicine has several clinical uses that are advantageous. Myocardial perfusion imaging, tumour imaging, bone metabolism imaging, regional and relative renal function detection, and brain imaging to detect and locate epileptic foci and determine the activity of brain tumours are only a few examples.
\par
The tracer travels to the location of the body of interest once the radiopharmaceutical is administered, which is usually by intravenous injection. Different radiopharmaceuticals will target various tissues.
A radiopharmaceutical is made up of a radionuclide and a pharmaceutical component. The radionuclide component of the tracer produces radiation in the form of gamma rays, beta particles, or positrons. 99mTc and 123I, for example, mostly release gamma rays, whereas 18F emits positrons. These tracers have short half-lives (99mTc and 18F have half-lives of 6 and 2 hours, respectively), allowing adequate time to complete the investigation while minimising patient radiation exposure.The pharmaceutical component of the tracer determines where the radiopharmaceutical will accumulate in the body. MDP (99mTc medronate) concentrates in the skeleton, whereas DMSA (99mTc dimercaptosuccinic acid) concentrates in the kidneys. 18F fluorodeoxyglucose (FDG) is a glucose analogue that is metabolised similarly to glucose in the body. In some circumstances, such as 99mTc mercaptoacetyltriglycine (MAG3) kidney and 99mTc-labeled red blood cells studies, imaging occurs right after the radiotracer is administered.Other radiotracers, such as 99mTc DMSA, 99mTc MDP, and FDG, require waiting periods (ranging from 1 to 3 hours) between the time the tracer is administered and the time the pictures are taken to allow the agent to disperse to the tissue of interest and clear from the blood or nearby tissues. Initial imaging and further imaging may be required in some circumstances, such as a 99mTc MDP bone scan.
\par
The scans can be dynamic, which means they track the amount of radioactivity in the tissue over time (e.g., an image every minute for 30 min). During this time interval, several frames are recorded that can be shown as a video recording of the movement. The acquisition of a "renogram" using the radiopharmaceutical 99mTc MAG3 is an example of a dynamic scan. An image of the agent in the kidneys is captured every minute or so for around 30 minutes in this study. The early images, or "frames," depict the agent's perfusion and uptake by the kidneys, whereas the later frames depict the rate of clearance from the kidneys.
\par
Nuclear medicine acquisitions can also be "gated" on an ECG, which is quite helpful for cardiac imaging. Gated cardiac scans, like dynamic scans, gather numerous images, but each image is from a distinct part of the heart cycle. The images are then shown as a cine of the beating heart, allowing myocardial blood flow (or "perfusion") and dynamic cardiac parameters like the ejection fraction to be assessed.
\par
The patient must be appropriately positioned to produce high-quality photos. Because most nuclear medicine imaging procedures necessitate the patient remaining still for an extended length of time, immobilisation measures to assist some patients, such as youngsters, in remaining still during imaging may be required. Depending on the patient's size, age, condition, and activities, sandbags, sticky tape, a papoose wrap with blanket, Velcro straps, and contoured pillows may be useful. If the patient is unable to remain motionless for the duration of the scan, sedation or even general anaesthesia may be required. Depending on the technique and the radiopharmaceutical being utilised, a nuclear medicine technologist uses a gamma camera or a positron emission tomography (PET)/CT scanner to acquire images.In many circumstances, a nurse is required during the acquisition process, especially when other drugs are involved in addition to the radiopharmaceutical. When a pharmaceutical technique is utilised to stress the heart, as well as when sedation or anaesthesia is used, are examples.
\par
Planar imaging, the most popular nuclear medicine technique, uses a gamma camera to provide a two-dimensional picture of where the radiopharmaceutical is located within the patient. Planar imaging is used in a variety of diagnostic procedures, including gastroenterology, orthopaedics, endocrinology, urology, and cancer. Planar imaging is appropriate for paediatric imaging since each image takes only a few minutes.The fundamental parts of a gamma camera are the collimator, the scintillator, a variety of photomultiplier tubes, and a PC. Gamma beams are transmitted from a radiopharmaceutical situated inside the patient's objective design or tissue. For instance, 99mTc MDP is a phosphate compound that specifically amasses in developing bone. The colli-mator is made of a gamma beam engrossing metal, like lead, and just permits those beams going in the appropriate
heading to be engaged with the imaging system. The least difficult type of collimation utilized in atomic medication imaging is the pinhole collimator, which comprises of a solitary opening or "pinhole" situated a ways off from the location material. This kind of collimation can give optically amplified pictures, which can be extremely valuable while imaging tiny patients or little constructions inside a bigger patient.
The most ordinarily utilized collimators are multihole, equal collimators. These collimators are intended for explicit radionuclides and imaging assignments; subsequently, an atomic medication center will have an assortment of explicit collimators. Models are low-energy, high-goal or medium-energy, universally useful collimators. The scintillator is the material touchy to gamma radiation and ordinarily comprises of a gem made of sodium io-dide that can change over the assimilated gamma beams into apparent light. The photomultiplier tubes retain the apparent light and convert it to a quantifiable electrical flow. Inside the PC, this electrical sign is changed over to an advanced picture or series of pictures on account of a dynamic or gated study


\end{document}